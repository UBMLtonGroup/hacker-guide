Backend takes input from SSA2, transforms it to RSSA, along with which
it instantiates object representation, analyzes liveness, decides
allocation points and inserts GC checks. Finally, it converts code
into machine representation.

\subsection{The RSSA Definition}
The RSSA language is defined in \tt{backend/rssa.sig}. A program
consists of multiple functions, each function has several
blocks. Controls go from one block to the other via a specially
defined transfer. Within each block, there are multiple
statements. Each statement performs a specific operation on
operands. The operands are commented as follows:

\begin{verbatim}
 structure Operand:
    sig
       datatype t =
          (* Accesses array fields *)
          ArrayOffset of {base: t,          (* Array base pointer *)
                          index: t,         (* Index of object    *)
                          offset: Bytes.t,  (* Byte offset of object *)
                          scale: Scale.t,   (* Size of element *)
                          ty: Type.t}       (* type of element *)
          (* Converts from one type to the other *)
        | Cast of t * Type.t
        | Const of Const.t
          (* EnsuresBytesFree is a pseudo-op used by C functions (like
           * GC_allocateArray) that take a number of bytes as an argument
           * and ensure that that number of bytes is free upon return.
           * EnsuresBytesFree is replaced by the limit check pass with
           * a real operand.
           *)
        | EnsuresBytesFree
          (* Access to GC state *)
        | GCState
          (* Object fields access *)
        | Offset of {base: t,
                     offset: Bytes.t,
                     ty: Type.t}
          (* Type constructor tag *)
        | ObjptrTycon of ObjptrTycon.t
          (* Runtime parameters in GCState *)
        | Runtime of Runtime.GCField.t
          (* A variable *)
        | Var of {ty: Type.t,
                  var: Var.t}
        (* ... *)
     end
\end{verbatim}

The operations in statement is commented in the follows:

\begin{verbatim}
structure Statement:
   sig
      datatype t =
         (* Bind a value to a variable: var x = ... *)
         Bind of {dst: Var.t * Type.t,
                  isMutable: bool,
                  src: Operand.t}
         (* Copy value of one operand to the other *)
       | Move of {dst: Operand.t,
                  src: Operand.t}
         (* Creating object *)
       | Object of {dst: Var.t * Type.t,
                    header: word,
                    size: Bytes.t (* including header *)}
         (* Apply primitive operators defined in atoms.fun *)
       | PrimApp of {args: Operand.t vector,
                     dst: (Var.t * Type.t) option,
                     prim: Type.t Prim.t}
         (* Profiling? *)
       | Profile of ProfileExp.t
         (* Profiling label? *)
       | ProfileLabel of ProfileLabel.t
       | SetExnStackLocal
       | SetExnStackSlot
       | SetHandler of Label.t (* label must be of Handler kind. *)
         (* ExnStack is as a slot under the call stack *)
       | SetSlotExnStack
   end
\end{verbatim}


\begin{verbatim}
structure Transfer:
   sig
      datatype t =
         (* Transfers by performing certain arithmetic operation *)
         Arith of {args: Operand.t vector,
                   dst: Var.t,
                   overflow: Label.t, (* Must be nullary. *)
                   prim: Type.t Prim.t,
                   success: Label.t, (* Must be nullary. *)
                   ty: Type.t}
         (* Call into C function *)
       | CCall of {args: Operand.t vector,
                   (* The type representing C function, can be
                    * tagged as mayGC etc *)
                   func: Type.t CFunction.t,
                   (* return is NONE iff the CFunction doesn't return.
                    * Else, return must be SOME l, where l is of kind
                    * CReturn.  The return should be nullary if the C
                    * function returns void.  Else, it should be unary with
                    * a var of the appropriate type to accept the result.
                    *)
                   return: Label.t option}
         (* Call into other functions *)
       | Call of {args: Operand.t vector,
                  func: Func.t,
                  return: Return.t}
         (* Jump *)
       | Goto of {args: Operand.t vector,
                  dst: Label.t}
       (* Raise implicitly raises to the caller.
        * I.E. the local handler stack must be empty.
        *)
       | Raise of Operand.t vector
         (* Return statement *)
       | Return of Operand.t vector
         (* Switch with type constructor pattern matching *)
       | Switch of Switch.t
   end
\end{verbatim}

\subsection{Packed Object Representation}
Before going into how backend works, we'll need to understand how
objects are represented in MLton. Historically, MLton has a flat
object representation, but it's removed in favor of a packed object
representation.

The file \tt{mlton/backend/packed-representation.fun} returns a
function that will convert SSA2 types into RSSA object
representations and at the same time, it generates necessary routines
for object tagging and pattern matching.

In MLton, pointers are tagged with two bits to distinguish them from
integers. MLton also distinguishes object pointers from non-object
pointers (e.g. pointers to a single integer). For non-object pointers,
MLton introduces small and big pointers. Small refers to addresses
that can be represented in a single integer, while Big refers to those
that cannot. This part of code is listed below:

\begin{verbatim}
  case rep of
     Rep.NonObjptr =>
        let
           val i = Bits.toInt (Type.width ty)
        in
           if i >= objptrBitsAsInt ()
              then makeBig ()
           else
              let
                 val {component, selects} =
                    case tr of
                       TupleRep.Direct z => z
                     | TupleRep.Indirect _ =>
                          Error.bug "PackedRepresentation.TyconRep.make: small Indirect"
                 val () = IntInf.inc numSmall
                 val () =
                    Array.update
                    (small, i,
                     {component = component,
                      con = con,
                      objptrTycon = objptrTycon,
                      selects = selects}
                     :: Array.sub (small, i))
              in
                 ()
              end
        end
   | Rep.Objptr _ => makeBig ()
\end{verbatim}

Here the objptrBitsAsInt is a query into runtime constants:

\begin{verbatim}
 val objptrBytes = Runtime.objptrSize
 val objptrBits = Promise.lazy (fn () => Bytes.toBits (objptrBytes ()))
 val objptrBitsAsInt = Promise.lazy (fn () => Bits.toInt (objptrBits ()))
\end{verbatim}

The most important part of packed-representation is called
``\tt{TupleRep}''. It basically summerizes they way object components
glue together. Cited from the comments:

\begin{verbatim}
TupleRep.make decides how to layout a sequence of types in an object,
or in the case of a vector, in a vector element.  Vectors are treated
slightly specially because we don't require element widths to be a
multiple of the word32 size.  At the front of the object, we place all
the word64s, followed by all the word32s.  Then, we pack in all the
types that are smaller than a word32.  This is done by packing in a
sequence of words, greedily, starting with the largest type and moving
to the smallest.  We pad to ensure that a value never crosses a word32
boundary.  Finally, if there are any objptrs, they go at the end of
the object. There is some extra logic here to specially represent
(boxed) tuples that are entirely comprised of primitive types.  The
primary motivation is that "word8 ref" and "word16 ref" are FFI types,
and must have representations that are compatible with C.  In
particular, on a big-endian platform, such sub-word32 components must
be at the low byte offset (but high bit offset) of the containing
word32.
\end{verbatim}

The selects generally correspond to Switch statement in C. It shifts
object to a position and matches its type tag. Each representation has
its own ways of unpack, but they are fairly straightforward.

\subsection{Liveness}

\subsection{Allocation}

\subsection{Limit Check}
Limit check tests whether a GC is needed. Since MLton pre-allocates a
fixed-size stacks on the heap, it also checks whether a stack growth
is necessary. It ensures 2 criterias.

\begin{itemize}
\item At any allocation of b bytes, $\mbox{frontier} + b <=
  \mbox{base} + \mbox{heapSize}$
\item At entry to each function, $\mbox{stackTop} <=
  \mbox{stackLimit}$
\end{itemize}

To achieve this, the runtime needs to provide multiple operands: Frontier,
Limit, LimitPlusSlop, StackLimit and StackTop. These are displayed as
the following graph:

% TODO: Add graph to illustrate Limit, LimitPlusSlop, Frontier etc

\subsubsection{Limit Check Criteria}
There are three different kinds of checks inserted, depending on the
amount being allocated and whether or not the program uses signal
handlers.

\noindent 1. If $b <= \mbox{LIMIT_SLOP}$, then continue (don't GC) if
$\mbox{frontier} <= \mbox{limit}$.

The reason this works is that if $\mbox{frontier} <= \mbox{limit}$ and
$b <= \mbox{LIMIT_SLOP}$, then

\begin{align}
  \mbox{frontier} + b <&= \mbox{limit} + \mbox{LIMIT_SLOP}
                       &= \mbox{limitPlusSlop}
                       &= \mbox{base} + \mbox{heapSize}
\end{align}

This works even if the program uses signal handlers, which set
limit to zero, since $\mbox{frontier} <= 0$ will always be false.

\noindent 2. If $b > \mbox{LIMIT_SLOP}$ and if the program doesn't use
signal handlers, then continue (don't GC) if $b <=
\mbox{limitPlusSlop} - \mbox{frontier}$ The reason this works is that
the condition is equivalent to

\begin{equation}
  b + \mbox{frontier} <= \mbox{limitPlusSlop} = \mbox{base} +
  \mbox{heapSize}
\end{equation}

We write the condition the way we do instead of the more obvious way
because "$b + \mbox{frontier}$" may overflow, while
$\mbox{limitPlusSlop} - \mbox{frontier}$ can not, unless the program
uses signal handlers.

\noindent 3. If b > LIMIT_SLOP and if the program uses signal
handlers, then continue (don't GC) if $\mbox{limit} > 0$ and $b <=
\mbox{limitPlusSlop} - \mbox{frontier}$

This is like the previous case, except that because the program uses signal
handlers, the runtime may have set limit to zero to indicate that a
signal needs to be handled.  So, we first check that this is not
the case before continuing as in previous one.

Stack limit checks are completely orthogonal to heap checks, and are simply
inserted at the start of each function.

\subsubsection{Limit Check Algorithm Reasoning}
Practically, it's not efficient to insert limit checks at each
entry of the block.

I'm including the original message that describes limit checks
\ref{url:limit-check}. Readers are encouraged to read following
discussions for better understanding of the algorithm.

\begin{verbatim}
Original Message
------------------------------------------------------------------
\end{verbatim}

Here's my latest attempt at formalizing a framework for limit check
insertion, leading to an approach for inserting limit checks based on
loop forests.  Let me know your thoughts (and please check my
proofs!).

\noindent\textbf{Input:}\\
\noindent A four-tuple

\begin{tabular}{|l|l|l|}
  \hline
  Label	&		   &	set of blocks\\\hline
  S	& Label -> P(Label)&	successors of a block\\\hline
  E	& P(Label)	   &	entry labels\\\hline
  A	& Label -> Nat	   &	number of bytes allocated in block\\\hline
\end{tabular}

\noindent\textbf{Output (limit check insertion):}\\
\noindent A two tuple

\begin{tabular}{|l|l|l|}
  \hline
  C	& Label -> Nat		& limit check amount (beginning of block)\\\hline
  F	& Label -> Nat		& free amount required before block\\
  \hline
\end{tabular}

\begin{verbatim}
Definition [path, starts]:
A path in (L, S) is a sequence of labels l0, l1, ... such that for all
i >= 0, l_i+1 in S(li).  p starts at l0.

Definition [V]:
V: Path x Int -> Int, defined by induction on the length of the path.

  V ([], n) = n
  V (l :: p, n) = V (p, max (n, C(l)) - A(l))

Intuitively, V (p, n) gives the amount free at the end of the path,
given that n is free at the beginning.

Definition [sound]:
The insertion is sound if for all l in E, for all p starting at l,
V (p, 0) >= 0.

Soundness depends on C, but not on F.

Definition [safe]:
The insertion is safe if
1. For all l in E, F(l) = 0.
2. For all l in Label, max(F(l), C(l)) - A(l) >= 0.
3. For all l in Label, for all l' in S(l), max(F(l), C(l)) - A(l) >= F(l').

Safety depends on both C and F.

Theorem:
If the insertion is safe, then for all l, for all p starting at l,
if n >= F(l), then V (p, n) >= 0.

Proof: by induction on the length of the path.
Base:
 V ([], n) = n >= F(l) >= 0.
   where the first inequality follows by assumption and the second by
   the range of F.
 V ([l], n) = max (n, C(l)) - A(l) >= max (F(l), C(l)) - A(l) >= 0
   where the first inequality follows by the assumption n >= F(l) and
   the second follows from condition 2 in the definition of safety.
Ind:
  V (l :: l' :: p, n) = V (l' :: p, max (n, C(l)) - A(l))
Notice that
     max (n, C(l)) - A(l)
  >= max (F(l), C(l)) - A(l)
  >= F(l')
where the first inequality follows from the assumption that n >= F(l)
and the second follows from condition 3 in the definition of safety.
Hence we can apply the induction hypothesis to conclude V (l' :: p,
max (n, C(l)) - A(l)) >= 0.
QED

Corrollary [safe => sound]:
If the insertion is safe, then it is sound.

Proof:
Let l in E and p starting at l be given. By condition 1 in the
definition of safety F(l) = 0.  Since 0 >= F(l), the above theorem
applies.  Hence, V (p, 0) >= 0.

The rest of this document is about producing a safe insertion.

Definition [cycle]:
A cycle is a path whose start and end labels are the same.

Definition [decycles]:
A set of labels L decycles the input if every cycle contains a label
in L.

Henceforth, assume L decycles the input.  We will place limit checks
at labels in E U L that need them.

Define the following function inductively

	R: Label -> Nat
	R(l) = A(l) + max { R(l') | l' in S(l) - (E U L)}

R is well defined because L decycles the input.

Define C and F by
	C(l) = if l in E U L
	       then R(l)
	       else 0
	F(l) = if l in E U L
               then 0
               else R(l)

Notice that by definition, for all l in Label, max(F(l), C(l)) = R(l).

Theorem: (C, F) is safe.
Proof:
1. For l in E, F(l) = 0 by the definition of F.
2. For all l in Label
	max(F(l), C(l)) - A(l)
	= R(l) - A(l)
	= max { R(l') | l' in S(l) - (E U L)}
	>= 0
3. For all l in Label, for all l' in S(l),
   	max(F(l), C(l)) - A(l)
        = R(l) - A(l)
        = max { R(l'') | l'' in S(l) - (E U L)}
   If l' in E U L, then F(l') = 0
        >= 0
   otherwise F(l') = R(l')
	>= R(l')
QED

So, all we need is an L that decycles the input, and we automatically
get a sound insertion.  One can use loop forests to build such an L.
I won't repeat the definitions from page 3 of the Ramalingam paper,
but there he defines loop forests, and shows that taking L to be the
set of loop headers decycles the input.  That gives one way of doing
limit check insertion.  It's not great because it still moves limit
checks forward into loops.  I propose to do the following.

Define two nodes to be in the same loop if they appear in the same
notInLoop vector in our representation of loop forests.  Basically,
that means that deepest loop that each of the nodes appears in is the
same.  Then define L by

	L = Headers U { l | exists l' not in the same loop as l with l in S(l') }

This will place limit checks at loop headers and post-exits, i.e. at
nodes that will prevent limit checks being pushed across loop
boundaries.
\end{verbatim}



For limit checks, when the fixed size allocations that follow a variable size
allocation are stored in the ``bytesAllocated'' field of the limitCheck field in
the AllocateArray variant of machine-output.sig
