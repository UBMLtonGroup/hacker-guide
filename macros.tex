\newcommand{\absolutelink}[1]{\link{\makeurl{#1}}}
\newcommand{\addr}{mlton.org}
\newcommand{\chaplab}[1]{\label{chapter:#1}}
\newcommand{\chapref}[1]{Chapter~\ref{chapter:#1}}
\newcommand{\chap}[2]{\chapter{#1}{\chaplab{#2}}}
\newcommand{\code}[1]{\htmladdnormallink{{\tt #1}}{../../#1}}
\newcommand{\deflab}[1]{\label{definition:#1}}
\newcommand{\defref}[1]{Definition~\ref{definition:#1}}
\newcommand{\figBegin}{\begin{figure}\begin{boxit}}
\newcommand{\figEnd}[2]{\caption{#1}\figlab{#2}\end{boxit}\end{figure}}
\newcommand{\figlab}[1]{\label{figure:#1}}
\newcommand{\figref}[1]{Figure~\ref{figure:#1}}
\newcommand{\lemlab}[1]{\label{lemma:#1}}
\newcommand{\lemref}[1]{Lemma~\ref{lemma:#1}}
\newcommand{\link}[1]{\htmladdnormallink{{\tt #1}}{#1}}
\newcommand{\mailto}[1]{\htmladdnormallink{{\tt #1}}{mailto:#1}}
\newcommand{\makeurl}[1]{http://www.\addr/MLton/#1}
\newcommand{\mltonmail}{\mailto{MLton@\addr}}
\newcommand{\mlton}{MLton}
\newcommand{\mosml}{Moscow ML}
\newcommand{\place}[1]{\item[\tt #1]\hspace{1in}\\}
\newcommand{\seclab}[1]{\label{section:#1}}
\newcommand{\secref}[1]{Section~\ref{section:#1}}
\newcommand{\smlnj}{SML/NJ}
\newcommand{\subsecc}[2]{\subsubsection{#1}\seclab{#2}}
\newcommand{\subsec}[2]{\subsection{#1}\seclab{#2}}
\newcommand{\subsubsec}[2]{\subsubsection{#1}\label{section:#2}}
\newcommand{\tabBegin}{\begin{table}}
\newcommand{\tablab}[1]{\label{table:#1}}
\newcommand{\tabref}[1]{Table~\ref{table:#1}}
\newcommand{\thmlab}[1]{\label{theorem:#1}}
\newcommand{\thmref}[1]{Theorem~\ref{theorem:#1}}
\newcommand{\userguide}{{\mlton} User Guide}
\newcommand{\version}{VERSION}
\renewcommand{\sec}[2]{\section{#1}\seclab{#2}}
\newenvironment{boxit}{\vbox\bgroup\hrule\hbox\bgroup\vrule\kern3pt\vbox\bgroup\kern3pt\advance\hsize by -6.8pt\relax}{\par\kern3pt\egroup\kern3pt\vrule\egroup\hrule\egroup}

\definecolor{mygreen}{rgb}{0,0.6,0}
\definecolor{mygray}{rgb}{0.5,0.5,0.5}
\definecolor{mymauve}{rgb}{0.58,0,0.82}
\definecolor{mypurple}{rgb}{0.627,0.126,0.941}
\lstset{numbers=left, numbersep=5pt, numberstyle=\color{mygray},
        frame=leftline, aboveskip=3em, belowskip=3em,
        xleftmargin=.25in, xrightmargin=.25in}
\lstset{
  language=ML,
  basicstyle=\small,
  breaklines=true,
  identifierstyle=\ttfamily,
  keywordstyle=\color[rgb]{0,0,1},
  commentstyle=\color[rgb]{0.133,0.545,0.133},
  stringstyle=\color{mypurple},
  otherkeywords={before,action,after,result,class}
}

\newcommand{\cfrag}[1]{
   \begin{minipage}{\linewidth}
   \lstset{language=C}
   \begin{lstlisting}
   #1
   \end{lstlisting}
   \end{minipage}
}

\newcommand{\cfragEnd}{\end{lstlisting}\end{minipage}}

\newcommand{\cfragBegin}{
   \begin{minipage}{\linewidth}
   \lstset{language=C}
   \begin{lstlisting}
}

